\newpage
\section{Leakage \& BCAs in the California Power Industry}

\subsection{A Perfectly Competitive Model of Leakage}

Here we consider a perfectly competitive model of emissions leakage in the Western Interconnection based largely off of \cite{fowlie2021border}. We begin constructing a model where there is no emissions pricing scheme and then impose an emissions tax. Alongside the emissions tax, we consider different possible border carbon adjustments with the potential to reduce leakage. 

This is not a model of residential electric power markets. Residential electric power prices involve complicated tier systems, connection charges, and many additional fees that distort the true price. In addition to discouraging electrification \citep[see][]{borenstein2021designing}, these structures make simulating these markets impractical in this analysis. Instead, we model the wholesale day-ahead electricity market. 

The Federal Energy Regulatory Commission's \emph{Energy Primer} \citep{ferc2020} describes how power grid operators and power plants decide what generation will take place. Power grid operators have the goal of ensuring reliable power supply at the least cost. This process plays out in two stages. In the first stage, operators prepare for forecasted demand by committing generators a day before. This is the day-ahead market, where power plants sell their generation in anticipation for the next day. In the second stage, operators make adjustments in real-time by calling on additional generators to either turn on or off through a process called Automatic Generation Control. Grid operators determine whether or not to increase generation by measuring the frequency (cycles/second) of the AC power lines. In the US, operators aim for a frequency of 60 Hz, so a frequency above or below 60 Hz means that the too much or too little generation respectively. 

The structure of these day-ahead markets varies considerably across the country. One important factor in identifying the structure of these markets is the existence of a Regional Transmission Operator (RTO) or an Independent System Operator (ISO). RTOs and ISOs are essentially two different names for the same entity.\footnote{\cite{rff_podcast1} says ``At one point, FERC---the Federal Energy Regulatory Commission---made a distinction between these entities. It's a distinction without a difference at this point."} These entities are non-profit organizations that have the primary responsibility of operating markets for electric power
across many generators (the power plants, i.e. the sellers of electricity) and load-serving organizations (utilities, large industrial users, i.e. the buyers of electricity). \cite{ferc2020} gives an excellent description of how these day-ahead markets operate under an RTO or ISO.
\begin{quote}
In day-ahead markets, the schedules for supply and usage of energy are compiled hours ahead of the beginning of the operating day. The RTO/ISO then runs a computerized market model that matches buyers and sellers throughout the market footprint for each hour throughout the day. The model then evaluates the bids and offers of the participants, based on the power flows needed to move the electricity throughout the grid from generators to consumers. Additionally, the model must account for changing system capabilities that occur, based on weather and equipment outages, plus the rules and procedures that are used to ensure system reliability. The market rules dictate that generators submit supply offers and that loads submit demand bids to the RTO/ISO by a deadline that is typically in the morning of the day-ahead scheduling. Typically, 95 percent of all load is scheduled in the day-ahead market and the rest is scheduled in real-time.
\end{quote}
This means that the wholesale day-ahead market has many of the characteristic features of a competitive market. The good bought and sold is homogeneous, and buyers have little ability to differentiate between  sellers. There are thousands of power plants selling power in any RTO/ISO and dozens of utilities and industrial costumers buying it. Moreover RTOs/ISOs dispatch additional generating units based on their marginal costs, which is suggestive that supply in the market is allocated competitively. Given this market structure, it seems most sensible to model this market as perfectly competitive. 

% \subsubsection*{The General Model}

% Consider a model of the Western Interconnection. Denote the set of all power plants in the interconnection as $N$ and the subset of power plants located in California with $N_\text{Cal}$. The set of regions $R$ in the interconnection, each with its own market for electric power and its own generation. For now, assume these markets behave competitively. Further, assume that plants in each regional market face linear inverse demand curves with the form
% \begin{equation}
% 	p_r = \alpha_r - \beta_r Q_r
% \end{equation}
% where $p_r$ is the price in region $r$ (\$/MWh), $Q_r$ is the total quantity of power demanded in region $r$ (MWh), and $\alpha_r$ and $\beta_r$ are regional constants. 

% Each power plant decides how much power to generate for each regional market. Let $q_{ir}$ denote the power in MWh that plant $i$ generates to sell in regional market $r$. Any plant can generate power for any other region, regardless of what region the plant is located in. 

% We can group all power plants into one of two groups: fossil fuel plants and non-fossil fuel plants. Fossil fuel plants each have a plant specific marginal cost that depends on some of the physical features of its generators and the cost of the fuel it uses (natural gas, coal, or occasionally, oil). Fossil fuel plants choose how much power to generate based on their marginal costs and the prices in each market. 

% Non-fossil fuel plants do not have such clear marginal costs. For instance, a windfarm does not decide how much power to generate based on its marginal cost, but is constrained the generate whatever the environment allows it to. We make the assumption that the total generation for any individual non-fossil fuel plant is fixed, and they have not marginal cost. Note that although the total generation of these plants is fixed, the allocation of their generation is not. These plants can still choose how much of their power to sell in each region. 

% While we might usually endow individual firms with behaviors (objective functions) and try to solve analytically for their aggregate outcome, here we endow the entire interconnection with a behavior instead. Standard economic theory says that the perfectly competitive outcome uniquely maximizes total surplus. Then to simulate this market, we need to find what generation and allocation from each plant maximizes the sum of the areas under the demand curve in each regional market minus any marginal costs, including costs from congestion, carbon taxes, and BCAs. The area under the demand curve in region $r$ is a trapezoid with area
% $$\frac12 \left[\alpha + \left( \alpha_r - \beta_r Q_r^*\right)\right] Q_r^* = \frac12 \left( 2\alpha_r - \beta_r \sum_{i \in N} q_{ir} \right) \sum_{i \in N} q_{ir}$$
% where $Q_r^*$ is the equilibrium quantity regional market $r$. This leads the the optimization problem:
% \begin{align*}
% \max_{\{q_{ir}\}} \hspace{2em}\left\{\sum_{r\in R} \left[ \frac12 \left( 2\alpha_r - \beta_r \sum_{i \in N} q_{ir} \right) \sum_{i \in N} q_{ir}\right] - \left(\sum_{(i, r) \in N \times R} c_i q_{ir}\right) - \tau \left(\sum_{(i, r) \in N \times R} \widetilde{e_{ir}} q_{ir}\right)\right\}
% \end{align*}
% where $\tau$ is the emissions tax (\$/ton of CO$_2$e) and $\widetilde{e_{ir}}$ is the assessed emissions intensity. This assessed emissions intensity is the emissions intensity that policies tax at. As we lay out next, each policy functions by changing what the assessed emissions intensity is for each plant-region pair. The first term in the objective function is the sum of the areas under the demand curves. The second term is the total total marginal operating cost, and the third term is the total emissions cost. The optimization problem is subject to the following constraints:
% \begin{gather*}
% 	q_{ir} \geq 0, ~\text{for all plants} ~i~ \text{and regions} ~r\\
% 	\sum_{r \in R} q_{ir} \leq q_i^\text{max}, ~\text{for all fossil fuel plants} ~i\\
% 	\sum_{r \in R} q_{ir} = \overline{q}_i, ~\text{for all non-fossil fuel plants} ~i\\
% 	\left(\sum_{i \in N_{r}} q_{is}\right)  + \left(\sum_{i \in N_s} q_{ir}\right) \leq T_{(r,s)}, ~\text{for all}~(r,s) \in N\times N ~\text{such that}~ r \neq s
% \end{gather*}
% The first constraint guarantees that plants do not try to generate negative power for a region---they can only generate power at their own plant. The second constraint restricts fossil fuel power plants to their maximum generation (i.e., their nameplate capacity). Non-fossil fuel power plants do not have marginal costs, so the third constraint set the total generation of a non-fossil fuel plant to an exogenous quantity. The final constraint is the transmission constraint. This says that for any two distinct regions $r$ and $s$, the sum of their exports to the other region cannot exceed the maximum transmission load of the lines between them $T_{(r,s)}$. If there were no transmission constraints, nearly total leakage would be  practically guaranteed as trade is completely exposed.

% \begin{table}
% \caption{Notation Summary}
% \centering
% \begin{tabular}{c p{10cm}}
% \hline\hline
% Notation & Description \\
% \hline
% 	$N$ & Set of power plants\\
% 	$R$ & Set of regions\\
% 	$i$ & Subscript denoting an arbitrary power plant, $\i \in N$\\
% 	$r$ & Subscript denoting an arbitrary region, $r \in R$\\
% 	$q_{ir}$ & Plant $i$'s generation to sell in region $r$ (MWh)\\
% 	$q_i^\text{max}$ & Fossil fuel plant $i$'s exogenously determined maximum generation (MWh)\\
% 	$\overline{q}_i$ & Non-fossil fuel plant $i$'s exogenously determined total generation (MWh)\\
% 	$p_r$ & Market price in region $r$\\
% 	$Q_r$ & Generation to sell in region $r$, $Q_r = \sum_{i=1}^N q_{ir}$\\
% 	$e_i$ & Actual emissions intensity of plant $i$ (tons CO$_2$e)\\
% 	$\widetilde{e_{ir}}$ & Assessed emission intensity; emissions intensity for tax purposes of plant $i$ selling in region $r$\\
% 	$d$ & Default emissions intensity\\
% \hline	\hline
% \end{tabular}
% \end{table}

% \subsubsection*{Scenario A: No Regulation}

% Under no regulation, policymakers do not assess emissions to any power plant. That is, $\widetilde{e_{ir}} = 0$.

% \subsubsection*{Scenario B: Complete Regulation}

% Under complete regulation, all plants pay a tax based on their actual emissions intensity in each market. Then set $\widetilde{e_{ir}} = e_i$ for each $i \in N$ and $r \in R$. 

% \subsubsection*{Scenario C: California Carbon Tax, No BCA}

% In this scenario, California implements a domestic carbon tax without a BCA. This means that all plants in California pay a tax based on their actual emissions intensity, but no other plants pay an emissions tax. Then set
% \[
% \widetilde{e_{ir}} = \begin{cases}
% 	~~e_i & \text{if}~~i\in N_\text{Cal}\\
% 	~~0 & \text{if}~~i\not\in N_\text{Cal}.
% \end{cases}
% \]

% \subsubsection*{Scenario D: California Carbon Tax, Uniform BCA}

% Once again, all plants in California face a domestic carbon tax and pay based on their actual emissions intensity. With the uniform BCA though, all plants outside of California pay an emissions tax on the power they export to California. Regardless of their actual emissions intensity, all plants exporting power to California face a default emissions intensity, $d$. In this scenario,
% \[
% \widetilde{e_{ir}} = \begin{cases}
% 	~~e_i & \text{if}~~i\in N_\text{Cal}\\
% 	~~d & \text{if}~~i\not\in N_\text{Cal} ~~\text{and}~~ r = \text{California}\\
% 	~~0 & \text{if}~~i\not\in N_\text{Cal} ~~\text{and}~~ r \neq \text{California}.
% \end{cases}
% \]

% \subsubsection*{Scenario E: California Carbon Tax, Differentiated BCA}

% Lastly, this scenario models a differentiated BCA. The assessed emissions intensities of all California plants is again their actual emissions intensity. Plants outside of California can choose to either pay their actual emissions intensity the default emissions rate, $d$. Assume that plants always choose the lower emissions intensity, so that clean plant with $e_i < d$ will choose to pay using their actual emissions rate and dirty plants with $e_i \geq d$ will choose to pay using the default emissions intensity. This means,
% \[
% \widetilde{e_{ir}} = \begin{cases}
% 	~~e_i & \text{if}~~i\in N_\text{Cal}\\
% 	~~e_i & \text{if}~~i\not\in N_\text{Cal} ~~\text{and}~~ r = \text{California} ~~\text{and}~~ e_i < d \\
% 	~~d & \text{if}~~i\not\in N_\text{Cal} ~~\text{and}~~ r = \text{California} ~~\text{and}~~ e_i \geq d \\
% 	~~0 & \text{if}~~i\not\in N_\text{Cal} ~~\text{and}~~ r \neq \text{California}.
% \end{cases}
% \]


% \subsection{Data}

% Estimating the simulation requires data on all individual power plants, regional electric power markets, and transmission across the entire Western Interconnection. 

% Regional market data comes from US Energy Information Administration's (EIA) Hourly Electric Grid Monitor from 2019. I pull daily generation data by region and by fuel type for the Western Electric Coordinating Council, which governs the Western Interconnection. This aggregates the Western Interconnection into three regions: California, the Northwest, and the Southwest. Additionally, I pull price data from spot markets in each of these regions from the Inter-Continental Exchange (ICE) for the  247 days it is available in 2019. 

% Power plant data comes from the EPA's 2019 Emissions \& Generation Resource Integrated Database (\href{https://www.epa.gov/egrid/download-data}{eGRID}). This database contains annual summary statistics for each power plant in the US, including valuable information on annual emissions, generation, nameplate capacity (maximum generation), location, and fuel type. 


% \subsection{Simulation}

% To simulate the model, we use the CVXR package in R to solve the optimization problem for each day with data in 2019 under each of the different scenarios. 

% Before we can solve these optimization problems, we first need to calibrate several important parameters and exogenous variables. Following the approach of \cite{fowlie2021border}, I choose to set an elasticity of 0.075 for the demand curve at the observed price and quantity in the regional market, and then solve for the values of $\alpha_r$ and $\beta_r$ so that the demand curve each day will pass through the observed price and quantity. Future work will take a more rigorous approach to estimating these daily demand curves. We do not have the data to say what the individual generation of each non-fossil fuel plant was on each day. We do have regional data on the generation by fuel type though, so to make up for this, I fixed the daily generation of non-fossil fuel plants to be proportional to the total generation by the corresponding fuel type. The proportion for each plant is ratio of the nameplate capacity of the individual plant to the total nameplate capacity of all plants with the same fuel type in the region. For instance, suppose on a given day there was a total of 100 MWh generated from wind power in California. Then for a wind farm in California with 10\% of the total nameplate capacity for all wind farms in California, we set $\overline{q} = 10$. 

% I try to use values for the tax rate $\tau$ and the default emissions rate that are historically accurate. I set $\tau = \$19$ per ton, and set $d = 0.428$ tons per MWh. The next five tables compare the simulated annual generation from fossil fuel plants to the actual (observed) generation from these plants  over the same time period. For now, I use the simple regression:
% $$\text{Observed Generation} = \alpha + \beta \cdot \text{Simulated Generation}.$$
% If the simulation model matches the world reasonably well, then the simulated generation amounts should be similar to the observed generation amounts such that $\hat{\alpha} = 0$ and $\hat{\beta} = 1$. If $\hat{beta} > 1$, this indicates that simulated generation tends to be less than actual generation, and if $\hat{beta} < 1$, this indicates that simulated generation tends to be more than actual generation. We expect that the simulated policy that matches most closely to the actual policy (uniform BCAs, simulation D) is closest. 

% Tables 12 and 13 display the primary results of interest. In Table 12, we see the reduction in generation in each region by fuel type for each of the different policy simulations from the baseline scenario where there is no carbon pricing scheme. These results indicate substantial leakage when there is a carbon tax in California, but no BCA (Simulation C). California reduces its generation from natural gas substantially, as most of its power comes from natural gas. This results in large increases in both natual gas and coal generation in the Southwest. In Simulation E (Differentied BCA), we see that there is still substantial leakage as it appears low emissions intensity generation in the southwest is redirected to California. Uniform BCAs though appear much more successful in reducing emissions. These early results are consistent with our intuition of the apparent risks for reshuffling under differentiated BCAs. 

% \singlespacing
% \newpage
% \begin{table}[!htbp] \centering 
%   \caption{Annual Generation from Simulation A} 
% \begin{tabular}{@{\extracolsep{5pt}}lccc} 
% \hline \hline \\[-1.8ex]
%  & \multicolumn{3}{c}{Actual Generation (MWh)} \\ 
% \\[-1.8ex] & All Fossil Fuel & Natural Gas & Coal\\ 
% \hline \\[-1.8ex] 
%  Simulated Generation (MWh) & 1.035$^{***}$ & 0.769$^{***}$ & 1.192$^{***}$ \\ 
%   & (0.039) & (0.038) & (0.150) \\ 
%   & & & \\ 
%  Constant & 233,326.600$^{***}$ & 227,964.100$^{***}$ & 1,216,915.000$^{**}$ \\ 
%   & (51,713.200) & (40,188.380) & (459,448.200) \\ 
%   & & & \\ 
% Observations & 505 & 448 & 40 \\ 
% R$^{2}$ & 0.578 & 0.476 & 0.623 \\ 
% \hline \\[-1.8ex] 
% \textit{Notes:} & \multicolumn{3}{l}{$^{***}$Sig. at 1\% level, $^{**}$Sig. at 5\% level}, $^{*}$Sig. at 10\% level \\
% \end{tabular} 
% \end{table} 

% \begin{figure}
% 	\centering
% 	\caption{Annual Generation in Simulation A}
% 	\includegraphics[width=0.8\textwidth]{figures_4/p1A.png}
% \end{figure}



% \begin{table}[!htbp] \centering 
%   \caption{Annual Generation from Simulation B} 
% \begin{tabular}{@{\extracolsep{5pt}}lccc} 
% \hline \hline \\[-1.8ex] 
% & \multicolumn{3}{c}{Actual Generation (MWh)} \\ 
% \\[-1.8ex] & All Fossil Fuel & Natural Gas & Coal\\ 
% \hline \\[-1.8ex] 
%  Simulated Generation (MWh) & 0.980$^{***}$ & 0.775$^{***}$ & 1.499$^{***}$ \\ 
%   & (0.054) & (0.035) & (0.302) \\ 
%   & & & \\ 
%  Constant & 308,188.900$^{***}$ & 201,785.400$^{***}$ & 2,076,648.000$^{***}$ \\ 
%   & (61,984.660) & (38,309.220) & (536,768.500) \\ 
%   & & & \\ 
% Observations & 505 & 448 & 40 \\ 
% R$^{2}$ & 0.399 & 0.528 & 0.394 \\ 
% \hline \\[-1.8ex] 
% \textit{Notes:} & \multicolumn{3}{l}{$^{***}$Sig. at 1\% level, $^{**}$Sig. at 5\% level}, $^{*}$Sig. at 10\% level \\

% \end{tabular} 
% \end{table} 

% \begin{figure}
% 	\centering
% 	\caption{Annual Generation in Simulation B}
% 	\includegraphics[width=0.8\textwidth]{figures_4/p1B.png}
% \end{figure}


% \begin{table}[!htbp] \centering 
%     \caption{Annual Generation from Simulation C} 
% \begin{tabular}{@{\extracolsep{5pt}}lccc} 
% \hline \hline \\[-1.8ex] 
% & \multicolumn{3}{c}{Actual Generation (MWh)} \\ 
% \\[-1.8ex] & All Fossil Fuel & Natural Gas & Coal\\ 
% \hline \\[-1.8ex] 
%  Simulated Generation (MWh) & 1.107$^{***}$ & 0.853$^{***}$ & 1.247$^{***}$ \\ 
%   & (0.036) & (0.036) & (0.135) \\ 
%   & & & \\ 
%  Constant & 207,799.900$^{***}$ & 211,358.400$^{***}$ & 992,577.500$^{**}$ \\ 
%   & (46,820.290) & (37,065.500) & (422,863.100) \\ 
%   & & & \\ 
% Observations & 505 & 448 & 40 \\ 
% R$^{2}$ & 0.653 & 0.552 & 0.691 \\ 
% \hline \\[-1.8ex] 
% \textit{Notes:} & \multicolumn{3}{l}{$^{***}$Sig. at 1\% level, $^{**}$Sig. at 5\% level}, $^{*}$Sig. at 10\% level \\ 
% \end{tabular} 
% \end{table}

% \begin{figure}
% 	\centering
% 	\caption{Annual Generation in Simulation C}
% 	\includegraphics[width=0.8\textwidth]{figures_4/p1C.png}
% \end{figure}


% \begin{table}[!htbp] \centering 
%     \caption{Annual Generation from Simulation D} 
% \begin{tabular}{@{\extracolsep{5pt}}lccc} 
% \hline \hline \\[-1.8ex] 
% & \multicolumn{3}{c}{Actual Generation (MWh)} \\ 
% \\[-1.8ex] & All Fossil Fuel & Natural Gas & Coal\\ 
% \hline \\[-1.8ex] 
%  Simulated Generation (MWh) & 1.059$^{***}$ & 0.796$^{***}$ & 1.198$^{***}$ \\ 
%   & (0.039) & (0.039) & (0.149) \\ 
%   & & & \\ 
%  Constant & 233,730.300$^{***}$ & 228,613.700$^{***}$ & 1,192,854.000$^{**}$ \\ 
%   & (50,785.190) & (39,668.160) & (456,598.900) \\ 
%   & & & \\ 
% Observations & 505 & 448 & 40 \\ 
% R$^{2}$ & 0.592 & 0.488 & 0.629 \\ 
% \hline \\[-1.8ex] 
% \textit{Notes:} & \multicolumn{3}{l}{$^{***}$Sig. at 1\% level, $^{**}$Sig. at 5\% level}, $^{*}$Sig. at 10\% level \\
% \end{tabular} 
% \end{table}

% \begin{figure}
% 	\centering
% 	\caption{Annual Generation in Simulation D}
% 	\includegraphics[width=0.8\textwidth]{figures_4/p1D.png}
% \end{figure}


% \begin{table}[!htbp] \centering 
%   \caption{Annual Generation from Simulation E}
% \begin{tabular}{@{\extracolsep{5pt}}lccc} 
% \hline \hline \\[-1.8ex] 
% & \multicolumn{3}{c}{Actual Generation (MWh)} \\ 
% \\[-1.8ex] & All Fossil Fuel & Natural Gas & Coal\\ 
% \hline \\[-1.8ex] 
%  Simulated Generation (MWh) & 1.106$^{***}$ & 0.851$^{***}$ & 1.246$^{***}$ \\ 
%   & (0.036) & (0.037) & (0.136) \\ 
%   & & & \\ 
%  Constant & 208,861.600$^{***}$ & 212,208.400$^{***}$ & 999,794.800$^{**}$ \\ 
%   & (47,011.030) & (37,211.240) & (424,501.800) \\ 
%   & & & \\ 
% Observations & 505 & 448 & 40 \\ 
% R$^{2}$ & 0.650 & 0.549 & 0.689 \\ 
% \hline \\[-1.8ex] 
% \textit{Notes:} & \multicolumn{3}{l}{$^{***}$Sig. at 1\% level, $^{**}$Sig. at 5\% level}, $^{*}$Sig. at 10\% level \\
% \end{tabular} 
% \end{table} 

% \begin{figure}
% 	\centering
% 	\caption{Annual Generation in Simulation E}
% 	\includegraphics[width=0.8\textwidth]{figures_4/p1E.png}
% \end{figure}





% \begin{table}[!htbp] \centering 
%   \caption{Annual Generation, All Fossil Fuels} 
% \begin{tabular}{@{\extracolsep{5pt}}lccccc} 
% \hline \hline \\[-1.8ex]
%  & \multicolumn{5}{c}{Observed Generation (MWh)} \\ 
% \\[-1.8ex] & Simulation A & Simulation B & Simulation C & Simulation D & Simulation E\\ 
% \hline \\[-1.8ex] 
% Simulated & 1.035$^{***}$ & 0.980$^{***}$  & 1.107$^{***}$  & 1.059$^{***}$  & 1.106$^{***}$ \\ 
%  Generation (MWh)  & (0.039) & (0.054) & (0.036) & (0.039) & (0.036) \\
%   & & & & & \\ 
%  Constant & 233,327$^{***}$ & 308,189$^{***}$ & 207,800$^{***}$ & 233,730$^{***}$ & 208,862$^{***}$ \\ 
%   & (51,713) & (61,985) & (46,820) & (50,785) & (47,011) \\ 
%   & & & & & \\ 
% Observations & 505 & 505 & 505 & 505 & 505 \\ 
% R$^{2}$ & 0.578 & 0.399 & 0.653 & 0.592 & 0.650 \\ 
% \hline
% \textit{Notes:} & \multicolumn{5}{l}{$^{***}$Sig. at 1\% level, $^{**}$Sig. at 5\% level, $^{*}$Sig. at 10\% level}\\
% \end{tabular} 
% \end{table} 


% \begin{table}[!htbp] \centering 
%   \caption{Annual Generation, Natural Gas}
% \begin{tabular}{@{\extracolsep{5pt}}lccccc} 
% \hline \hline
% \\[-1.8ex] & \multicolumn{5}{c}{Observed Generation (MWh)} \\ 
% \\[-1.8ex] & Simulation A & Simulation B & Simulation C & Simulation D & Simulation E\\ 
% \hline \\[-1.8ex] 
% Simulated & 0.769$^{***}$ &  0.775$^{***}$ & 0.853$^{***}$ & 0.796$^{***}$ & 0.851$^{***}$ \\ 
%   Generation (MWh) & (0.038) & (0.035) &  (0.036)  & (0.039)  & (0.037) \\ 
%   & & & & & \\ 
%  Constant & 227,964$^{***}$ & 201,785$^{***}$ & 211,358$^{***}$ & 228,614$^{***}$ & 212,208$^{***}$ \\ 
%   & (40,188) & (38,309) & (37,066) & (39,668) & (37,211) \\ 
%   & & & & & \\ 
% Observations & 448 & 448 & 448 & 448 & 448 \\ 
% R$^{2}$ & 0.476 & 0.528 & 0.552 & 0.488 & 0.549 \\ 
% \hline 
% \textit{Notes:} & \multicolumn{5}{l}{$^{***}$Sig. at 1\% level, $^{**}$Sig. at 5\% level, $^{*}$Sig. at 10\% level }\\
% \end{tabular} 
% \end{table} 


% \begin{table}[!htbp] \centering 
%   \caption{Annual Generation, Coal}
% \begin{tabular}{@{\extracolsep{5pt}}lccccc} 
% \hline \hline
% \\[-1.8ex] & \multicolumn{5}{c}{Observed Generation (MWh)} \\ 
% \\[-1.8ex] & Simulation A & Simulation B & Simulation C & Simulation D & Simulation E\\ 
% \hline \\[-1.8ex] 
%  Simulated & 1.192$^{***}$ & 1.499$^{***}$ & 1.247$^{***}$  & 1.198$^{***}$ & 1.246$^{***}$ \\ 
%   Generation (MWh) & (0.150) & (0.302) & (0.135) & (0.149)  & (0.136) \\ 
%   & & & & & \\
%  Constant & 1,216,915$^{**}$ & 2,076,648$^{***}$ & 992,578$^{**}$ & 1,192,854$^{**}$ & 999,795$^{**}$ \\ 
%   & (459,448) & (536,769) & (422,863) & (456,599) & (424,502) \\ 
%   & & & & & \\ 
% Observations & 40 & 40 & 40 & 40 & 40 \\ 
% R$^{2}$ & 0.623 & 0.394 & 0.691 & 0.629 & 0.689 \\ 
% \hline
% \textit{Notes:} & \multicolumn{5}{l}{$^{***}$Sig. at 1\% level, $^{**}$Sig. at 5\% level, $^{*}$Sig. at 10\% level }\\
% \end{tabular} 
% \end{table} 


% \begin{table}[!htbp] \centering 
%   \caption{Regional Fossil Fuel Generation}
% \begin{tabular}{@{\extracolsep{5pt}} lcccc} 
% \hline \hline \\[-1.8ex] 
% & \multicolumn{4}{c}{Simulated Change in Generation from No Carbon Pricing (GWh)}\\
% Region \& Fuel & Simulation B & Simulation C & Simulation D & Simulation E \\ 
% \hline \\[-1.8ex] 
% \textit{California} \\
% \quad Coal & -3505 & -1 & 0 & -38 \\ 
% \quad Natural Gas & -1881 & -16551 & -6651 & -15958 \\ 
% \textit{Northwest}\\
% \quad Coal & -29632 & 1 & 124 & 2 \\ 
% \quad Natural Gas & 10016 & 134 & 427 & 135 \\
% \textit{Southwest}\\
% \quad Coal & -5354 & 3826 & 320 & 3726 \\ 
% \quad Natural Gas & 5630 & 8257 & 114 & 7651 \\ 
% \hline \\[-1.8ex] 
% \end{tabular} 
% \end{table} 



% \begin{table}[!htbp] \centering 
%   \caption{Regional CO$_2$e Emissions}
% \begin{tabular}{@{\extracolsep{5pt}} lcccc} 
% \hline \hline \\[-1.8ex] 
% & \multicolumn{4}{c}{Simulated Change in Emissions from No Carbon Pricing (Kilotonnes)}\\
% Region \& Emissions Source & Simulation B & Simulation C & Simulation D & Simulation E \\ 
% \hline \\[-1.8ex] 
% \textit{California} \\
% \quad Coal & $-$3437 & $-$1 & 0 & $-$37 \\ 
% \quad Natural Gas & $-$1100 & $-$8056 & $-$3373 & $-$7770 \\
% \textit{Northwest}\\
% \quad Coal & $-$34640 & 2 & 157 & 4 \\ 
% \quad Natural Gas & 4362 & 91 & 216 & 92 \\
% \textit{Southwest}\\
% \quad Coal & $-$5930 & 4327 & 355 & 4207 \\ 
% \quad Natural Gas & 2458 & 3490 & 78 & 3212 \\ 
% Total & $-$38287	& $-$147 & $-$2567 & $-$292\\
% \hline \\[-1.8ex] 
% \end{tabular} 
% \end{table}