\newpage
\section*{Conclusion}
\addcontentsline{toc}{section}{Conclusion}

There is overwhelming evidence for anthropogenic climate change and evidence that climate change represents a significant threat to human welfare globally. Careful consideration of the issue has created broad support amongst economists and public policy scholars for climate policy that is capable of mitigating some of the damages from climate change by reducing greenhouse gas emissions. Although there are many strategies to do so, carbon pricing is one strategy that is particularly popular with economists and relatively popular around the world. 

A review of the literature on carbon pricing suggests that, by and large, carbon pricing is an important policy instrument for decarbonization. Carbon pricing devotees cite the cost-minimizing nature of the policy and the promise it offers to reduce emissions quickly. It is worth acknowledging though that carbon pricing is no cure all. Context is key, and contemporary research on carbon pricing has identified situations where other policies may be able to achieve decarbonization goals at a cost comparable to carbon pricing. 

A recent concern with carbon pricing is its effect on local environmental inequalities. Carbon pricing can redistribute emissions-intensive economic activity and redistribute the co-pollutants associated with this activity as well. Because carbon pricing policies do not contain any specific mechanisms that prevent the redistribution of co-pollutants towards communities that are already most affected by environmental degradation, many local environmental advocates have expressed concerns about the role of carbon pricing in climate policy design. This is especially true in California, where concerns about the implications of carbon pricing for environmental inequality almost ended the state's cap-and-trade program. 

The central goal of this text has been to address these concerns and identify the effect of carbon pricing on disparities in air pollution concentrations between disadvantaged communities and non-disadvantaged communities. I do so first be developing a model of carbon pricing and environmental inequality in the context of the Western US power grid. This model makes two key contributions. First, the model explicitly considers these air pollution disparities, whereas the most comparable models in the literature consider air pollution disparities only as an extension and do not give these disparities serious consideration. Second, the model generalizes the geography from other models into a multiregion model. This modification creates a familiar but unaddressed mechanism for carbon pricing to redistribute air pollution towards disadvantaged communities as a result of disparities in regulatory stringency. That is, this model is unique in that it considers the potential for unilateral carbon pricing to drive emissions-intensive economic activity into jurisdictions without a carbon price, and exacerbate environmental inequality in these jurisdictions. 

With data on power plants, electricity demand, transmission, and disadvantaged communities, I fit the model to make predictions about the effect of rising carbon prices on environmental inequality. The primary simulation results suggest that rising carbon prices in California will shift exacerbate existing inequalities in nitrogen oxide concentrations and have little-to-no effect of inequalities in sulfur dioxide and fine particulate matter concentrations. Further, the results suggest that exacerbated disparities in nitrogen oxide concentrations are due to the new mechanism highlighted in the model, demonstrating the importance of considering the incomplete nature of carbon pricing programs when estimating the impact of carbon pricing on environmental inequality. 
