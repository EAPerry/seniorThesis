\newpage
\section*{Conclusion}
\addcontentsline{toc}{section}{Conclusion}

In this final section, I revisit and summarize each chapter of the text and synthesize major ideas from each chapter to describe the implications of carbon pricing for environmental inequality. This text begins by discussing the climate science that motivates all research and policy related to climate change. In Chapter 1, a review of the climate science literature leads first to the conclusion that not only is climate change happening, but that current climate change is attributable to humans. Building on this, the section established that greenhouse gas emissions are the physical driver of climate change. These emissions come from the burning of fossil fuels, and on a per capita basis, the US emits more of these greenhouse gases than any other global superpower. In addition to the substantial scientific evidence asserting the existence of anthropogenic climate change, this section also reviews evidence on the impacts of climate change. The United Nations' Intergovernmental Panel on Climate Change has developed a set of Representative Key Risks that highlight the far reaching and hard hitting effects of climate change on natural life and human societies. Economists have their own framework for measuring the costs of climate change, and the best available evidence suggests that one metric ton of carbon dioxide emissions leads to present discounted climate damages of \$185---a distressing number when we consider that the US alone was responsible for 6.34 billion metric tons of carbon dioxide equivalent greenhouse gas emissions in 2021. This is all to say, climate change presents a serious threat to human society.

Analogous to the first chapter, Chapter 2 lays the economic foundation for climate change and climate policy. Before describing foundational concepts in climate economics, the chapter begins by motivating the use of economic analysis in climate policy decision making. The paper investigates whether or not climate policy is necessary to curb greenhouse gas emissions through an economic lense, introducing externalities, public goods, and alternatives to policy. This proceeds by considering differences in market-based policy instruments and command-and-control instruments, and describing the use of both within environmental and climate policy. Given the significance of carbon pricing throughout the paper, a separate section considers the basic theory behind carbon taxes and cap-and-trade programs in greater detail. Among other takeaways, this discussion finds that carbon pricing has merits over command-and-control policies for emissions reductions, but that these emissions reductions are highly contextual. While it appears that carbon pricing is generally still the ideal strategy for decarbonizing as quickly as possible, there are nuanced concerns related to the specific context of the policy that might mean there are exceptions to this rule. Finally, the chapter concludes by considering climate policy in a global context by reviewing literature on emissions leakage and border carbon adjustments.

While the first two chapters focus on describing climate change and global air pollutants, Chapter 3 brings the potential issues carbon pricing might present for the distribution of local air pollutants under inspection. Around the world, local (ambient) air pollution arguably presents a threat to human health and well being on par with climate change. Even domestically, local air pollutants continue to endanger substantial portions of the population. Greenhouse gases and local air pollutants are often emitted together, and as result, policy that changes the distribution of one type of pollutant can change the distribution of the other as well. Although economists widely favor a carbon price over alternative climate policies, little is known about how carbon pricing policies might redistribute local air pollution. This is of particular concern in California, where many residents have asserted that the state's cap-and-trade program has redistributed local air pollutants towards disadvantaged communities. The remainder of the text focuses on answering the question related to these concerns: do carbon pricing policies exacerbate existing inequalities in the concentrations of air pollutants? Related research focuses primarily on ex-post analysis of emissions sectors other than electric power generation. Apart from \cite{weber2021dynamic}, there are no other papers that build ex-ante models to consider how carbon pricing policies will redistribute local air pollution towards disadvantaged communities. 

Chapter 4 moves on this question by developing a novel economic model of environmental inequality associated with electric power generation. The model relies heavily on the model in \cite{weber2021dynamic}, but generalizes it into a multi-region model that can accommodate region-specific carbon prices and includes a measure of air pollution disparities based on a similar measure in \cite{hernandez2023environmental}. In the model, power plants make investment and operating decisions within perfectly competitive wholesale markets for electricity. Although the model is more computational than analytical in nature, the chapter concludes by informally characterizing the pathways through which carbon pricing could affect air pollution disparities. This discussion leads to the prediction that the air pollution concentrations of disadvantaged communities will rise relative to non-disadvantaged communities primarily under two scenarios: (1) if relatively clean power plants are disproportionately located near disadvantaged communities, or (2) if less regulated power plants are disproportionately located near the disadvantaged communities. While the first of these pathways is also apparent in \cite{weber2021dynamic}, the second of these pathways is unique to this paper. 

The final chapter of the text, Chapter 5, applies data to simulate the model from Chapter 4. The dataset used incorporates records from power plants across the Western Interconnection to simulate generation, investment, greenhouse gas emissions, local air pollutant emissions, and disparities in local air pollutant concentrations. Although these results are not robust in the sense that they provide conclusive and authoritative estimates of the disparities in air pollution concentrations, they are suggestive of several interesting and important results. Namely that, in the simulation's central estimates, increases in the carbon price exacerbate existing inequalities in NO$_x$ concentrations. Not only does the difference in the average concentration of NO$_x$ between disadvantaged and non-disadvantaged communities increase (i.e., the EI Gap increases),but the NO$_x$ concentration itself actually increases for disadvantaged communities. The other two air pollutants in the study, SO$_2$ and PM2.5, do not see any substantial reshuffling. Still, NO$_x$ concentration are the primary pollutant of concern, and this study provides suggestive evidence that carbon pricing can in fact, exacerbate envrionmental inequalities. Moreover, the redistribution of NO$_x$ concentrations towards disadvantaged communities appears to be a result of the second pathway from Chapter 4, a result driven by a mechanism unique to this model. 

% This paper leaves many open questions related to carbon pricing and the relationship between carbon pricing policies and environmental inequalities. Despite this,
Together, this research leads to three primary conclusions. First, a review of the literature and results from this study indicate that while carbon pricing appears to be an effective policy for decarbonization, it is not the panacea that economists have long made it out to be. In Chapter 2, \cite{borenstein2022carbon} found that in the electric power industry, carbon pricing is only negligibly more cost-effective than alternative clean energy standards (a command-and-control policy) and clean energy subsidies may actually be more effective at creating socially efficient electricity prices. So far, other policies appear to work well to reduce greenhouse gas emissions as well, though carbon pricing programs are increasingly important for decarbonization efforts. This study highlights the potential for carbon pricing programs to redistribute air pollution in unequal ways---a result that demonstrates the potential for carbon pricing to undermine related environmental goals related to environmental justice. There is not nearly enough evidence to suggest that carbon pricing is without a doubt disproportionately harmful to communities already overburdened with pollution and with more sensitive populations. It is clear that this is a serious concern that carbon pricing programs have yet to engage with in any legitimate way. 

Building off the previous conclusion, the second takeaway is that there is a need for additional research that can engage questions at the intersection of carbon pricing and envrionmental inequality. Central to this will be the ability of future models to incorporate more elaborate policies that, for instance, combine carbon pricing policies with local air pollution controls. Ensuring that the energy transition occurs in equitable ways is a central goal of policymakers. The results of this study add credibility to existing concerns about the potential for carbon pricing to lead to inequitable outcomes, and provides some intuition behind this. Rather than continuing to study the simplistic counterfactual of ``no carbon price,'' future work will need to consider more accurate policy counterfactuals, such as combinations of carbon pricing programs and local air pollution controls for a subset of communities. Such policies are likely the best bet for retaining the speed and efficacy of carbon pricing, while also prioritizing reductions in environmental inequalities. 

Finally, this study speaks to the importance of a broader geographic scope when considering the distributional implications of carbon pricing. Previous work has focused on examining the effect of carbon pricing on the distribution of outcomes only within the regulated jurisdiction. However, in this work, I consider also unregulated jurisdictions that are connected through trade. Chapter 2 discusses emissions leakage, a reality that implies that if any areas were to see increases in air pollutant concentrations, it would be communities in these less regulated jurisdictions. The simulation results demonstrate that this distinction truly matters. The EI Gap across the Western Interconnection increases by far more than the EI Gap in California alone, indicating that the most unequal effects of the policy are actually created by shifting air pollution towards disadvantaged communities outside of California. This result will be important to future work that considers the implications of carbon pricing for environmental inequality. 


% Chapter 1 lays the physical science foundation for climate change and climate policy. This starts by describing climate change and reviewing the evidence that not only is climate change happening, but that current climate change is attributable to humans. After establishing greenhouse gas emissions as the physical driver of climate change, the discussion shifts to contextualizing the composition and trends of greenhouse gas emissions. The chapter concludes with a discussion of the impacts of climate change. This discussion begins by walking through a set of eight major climate risks developed by the United Nations' Intergovernmental Panel on Climate Change (IPCC), and ends by describing an aggregated measure of the future climate impacts from emissions today that economists call the social cost of carbon. 

% Analogous to the first chapter, Chapter 2 lays the economic foundation for climate change and climate policy. Before describing foundational concepts in climate economics, the chapter begins by motivating the use of economic analysis in climate policy decision making. The paper investigates whether or not climate policy is necessary to curb greenhouse gas emissions through an economic lense, introducing externalities, public goods, and alternatives to policy. This proceeds by considering differences in market-based policy instruments and command-and-control instruments, and describing the use of both within environmental and climate policy. Given the significance of carbon pricing throughout the paper, a separate section considers the basic theory behind carbon taxes and cap-and-trade programs in greater detail. Finally, the chapter concludes by considering climate policy in a global context by reviewing literature on the pollution haven hypothesis, emissions leakage, and border carbon adjustments.


% Chapter 3


% Chapter 4 outlines a novel economic model of environmental inequality associated with electric power generation. The model relies heavily on the model in \cite{weber2021dynamic}, but generalizes it into a multi-region model that can accommodate region-specific carbon prices and includes a measure of air pollution disparities inspired by \cite{hernandez2023environmental}. It features generators that make investment and operating decisions within perfectly competitive wholesale markets for electricity. Although the model is more computational than analytical in nature, the chapter concludes by informally characterizing the pathways through which carbon pricing could affect air pollution disparities. \textbf{MORE ABOUT THE MODEL RESULTS HERE}

% Chapter 5 takes data to the model from Chapter 4. 

% The overarching goal of this paper is to allow these two components---climate policy and inequality---to speak in tandem with one another. Specifically, 




















% There is overwhelming evidence for anthropogenic climate change and evidence that climate change represents a significant threat to human welfare globally. Careful consideration of the issue has created broad support amongst economists and public policy scholars for climate policy that is capable of mitigating some of the damages from climate change by reducing greenhouse gas emissions. Although there are many strategies to do so, carbon pricing is one strategy that is particularly popular with economists and relatively popular around the world. 

% A review of the literature on carbon pricing suggests that, by and large, carbon pricing is an important policy instrument for decarbonization. Carbon pricing devotees cite the cost-minimizing nature of the policy and the promise it offers to reduce emissions quickly. It is worth acknowledging though that carbon pricing is no cure all. Context is key, and contemporary research on carbon pricing has identified situations where other policies may be able to achieve decarbonization goals at a cost comparable to carbon pricing. 

% A recent concern with carbon pricing is its effect on local environmental inequalities. Carbon pricing can redistribute emissions-intensive economic activity and redistribute the co-pollutants associated with this activity as well. Because carbon pricing policies do not contain any specific mechanisms that prevent the redistribution of co-pollutants towards communities that are already most affected by environmental degradation, many local environmental advocates have expressed concerns about the role of carbon pricing in climate policy design. This is especially true in California, where concerns about the implications of carbon pricing for environmental inequality almost ended the state's cap-and-trade program. 

% The central goal of this text has been to address these concerns and identify the effect of carbon pricing on disparities in air pollution concentrations between disadvantaged communities and non-disadvantaged communities. I do so first be developing a model of carbon pricing and environmental inequality in the context of the Western US power grid. This model makes two key contributions. First, the model explicitly considers these air pollution disparities, whereas the most comparable models in the literature consider air pollution disparities only as an extension and do not give these disparities serious consideration. Second, the model generalizes the geography from other models into a multi-region model. This modification creates a familiar but unaddressed mechanism for carbon pricing to redistribute air pollution towards disadvantaged communities as a result of disparities in regulatory stringency. That is, this model is unique in that it considers the potential for unilateral carbon pricing to drive emissions-intensive economic activity into jurisdictions without a carbon price, and exacerbate environmental inequality in these jurisdictions. 

% With data on power plants, electricity demand, transmission, and disadvantaged communities, I fit the model to make predictions about the effect of rising carbon prices on environmental inequality. The primary simulation results suggest that rising carbon prices in California will shift exacerbate existing inequalities in nitrogen oxide concentrations and have little-to-no effect of inequalities in sulfur dioxide and fine particulate matter concentrations. Further, the results suggest that exacerbated disparities in nitrogen oxide concentrations are due to the new mechanism highlighted in the model, demonstrating the importance of considering the incomplete nature of carbon pricing programs when estimating the impact of carbon pricing on environmental inequality. 
