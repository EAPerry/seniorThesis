~
% \newpage
% \section*{Acknowledgements}

\newpage

\begin{center}
	\Large \href{https://github.com/EAPerry/seniorThesis/raw/main/Writing/Draft/main.pdf}{Click Here for Latest Version of the Paper}\\
	\normalsize Version: \today
\end{center}


\section*{Preface}
\addcontentsline{toc}{section}{Preface}

Economists widely support the implementation of carbon pricing policies to reduce greenhouse gas emissions and mitigate the damages of climate change. The primary justification for carbon pricing policies, such as carbon taxes and cap-and-trade programs, is that they are efficient in the sense that they minimize the total cost of abatement. Alternative command-and-control policies could induce the same level of abatement, but they would likely come with a higher cost. Although this feature of environmental markets ensures the greatest net benefit for a given level of abatement, it does not make any guarantees about the distribution of these benefits. The distributional implications of carbon pricing schemes are generally ambiguous. 

Given that equity has become a cornerstone of the contemporary public discourse on climate change, it should not be surprising that leading voices within the environmental community remain skeptical of carbon pricing. Contemporary carbon pricing programs typically feature some form of a ``carbon dividend'' as a way to redistribute revenue generated through carbon pricing towards those who are most exposed to climate change risk. Many jurisdictions with carbon pricing schemes do redirect sizeable portions of the revenue generated through these programs into communities that already face the worst environmental degradation. These efforts 


may not warrant any serious concern as 



That is, because they do not have any mechanism to explicitly prevent redistribution of local air pollutants, they might incidentally reproduce inequitable environmental outcomes. For example, a carbon tax might cause a power plant in a high-income neighborhood to shut down entirely while a power plant in a low-income neighborhood picks

up its slack and maintains both its production and the associated air pollution, 


exacerbating disparities in air quality.


The objective of this paper is to analyze the implications of these carbon pricing schemes for environmental inequality. To do this, I focus on local air pollution from the electric power industry and study how carbon pricing in this industry affects the difference in air pollution concentrations between low- and high-socioeconomic status communities. 

This analysis falls into five chapters. The first two chapters provide a broad introduction to climate economics, with Chapter 1 reviewing the physical science of climate change and Chapter 2 reviewing the fundamental economic elements of climate policy design. A basic understanding of carbon pricing and emissions leakage are key to later work in the paper, so Chapter 2 takes a detailed look at both of these ideas. Using the foundational ideas from the first two chapters, Chapter 3 reviews the specific context of the , reviewing wholesale electricity markets, local air pollution, and the immediately adjacent literature. Chapter 4 outlines a novel economic model of environmental inequality associated with electric power generation. The model relies heavily on the model in \cite{weber2021dynamic}, but generalizes it into a multi-region model that can accommodate region-specific carbon prices and includes a measure of air pollution disparities inspired by \cite{hernandez2023environmental}. It features generators that make investment and operating decisions within perfectly competitive wholesale markets for electricity. Chapter 5 confronts the model from Chapter 4 with data by simulating generator investment and operating decisions over a three year period across the US Western Interconnection. I use the simulation model to measure how a carbon tax on electricity generation in California affects air pollution disparities. This involves creating the relevant counterfactual by simulating air pollution disparities with and without the carbon tax. \textbf{Specific results are forthcoming.}

% \begin{abstract}
% 	Economists widely support the implementation of a carbon pricing scheme (a carbon tax or cap-and-trade program) to mitigate the damages of climate change. Despite this broad support, the effect of carbon pricing on the distribution of outdoor air pollution and the resulting disparities in air pollution exposure between communities remains understudied. This research will concentrate on studying this concern in the context of a cap-and-trade program on California's electric power industry. Methodologically, I will expand on the work of Weber (2021) by (1) explicitly modeling the ``environmental justice gap'' from , (2) considering an open economy with the potential for the redistribution of generation outside of California, and (3) potentially employing a chemical transport model to more accurately estimate air pollution exposure. \textbf{Details of results are ~forthcoming~.}
% \end{abstract}


% Chapter 1 lays the physical science foundation for climate change and climate policy. This starts by describing climate change and reviewing the evidence that not only is climate change happening, but that current climate change is attributable to humans. After establishing greenhouse gas emissions as the physical driver of climate change, the discussion shifts to contextualizing the composition and trends of greenhouse gas emissions. The chapter concludes with a discussion of the impacts of climate change. This discussion begins by walking through a set of eight major climate risks developed by the United Nations' Intergovernmental Panel on Climate Change (IPCC), and ends by describing an aggregated measure of the future climate impacts from emissions today that economists call the social cost of carbon. 

% Analogous to the first chapter, Chapter 2 lays the economic foundation for climate change and climate policy. Before describing foundational concepts in climate economics, the chapter begins by motivating the use of economic analysis in climate policy decision making. The paper investigates whether or not climate policy is necessary to curb greenhouse gas emissions through an economic lense, introducing externalities, public goods, and alternatives to policy. This proceeds by considering differences in market-based policy instruments and command-and-control instruments, and describing the use of both within environmental and climate policy. Given the significance of carbon pricing throughout the paper, a separate section considers the basic theory behind carbon taxes and cap-and-trade programs in greater detail. Finally, the chapter concludes by considering climate policy in a global context by reviewing literature on the pollution haven hypothesis, emissions leakage, and border carbon adjustments.


% Chapter 3


% Chapter 4 outlines a novel economic model of environmental inequality associated with electric power generation. The model relies heavily on the model in \cite{weber2021dynamic}, but generalizes it into a multi-region model that can accommodate region-specific carbon prices and includes a measure of air pollution disparities inspired by \cite{hernandez2023environmental}. It features generators that make investment and operating decisions within perfectly competitive wholesale markets for electricity. Although the model is more computational than analytical in nature, the chapter concludes by informally characterizing the pathways through which carbon pricing could affect air pollution disparities. \textbf{MORE ABOUT THE MODEL RESULTS HERE}

% Chapter 5 takes data to the model from Chapter 4. 

% The overarching goal of this paper is to allow these two components---climate policy and inequality---to speak in tandem with one another. Specifically, 
