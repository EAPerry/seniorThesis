~
\newpage
\section*{Acknowledgements}

Although I sincerely hope that this text would prompt meaningful thought on topics at the intersection of climate policy and environmental inequalities, the ultimate goal of this senior thesis is to create a product that accurately encapsulates the skills I have been able to develop as a scholar as I approach life's next checkpoint. This product would be incomplete without any discussion of this development process---in particular, discussion of the people who have invested their time, effort, and resources to support my development. I believe that reflecting on and acknowledging these sacrifices is not just an exercise in giving thanks but is a vital component of the text, of equal or greater value to any other discussion that follows. 

First and foremost, I thank my family. I cannot imagine better parents than my own. My mother Kerri is the hardest working person I know, and 
I know that any ounce of perseverance I have is gift she has given me through her own example. My father Craig is unmatched in his humility and generosity. There is no Earthly reward for investing in others as earnestly as he does, and I hope that one day I can learn to replicate even a fraction of his generosity. My older sister Eryn is the original inspiration for everything I have ever wanted to do academically. She taught me to love learning and continues to inspire me with both her insight and her example.  
%Few people are as fortunate as I am to have a family that is so supportive and accepting, both personally and professionally. 

Second, I thank members of the Coe faculty for their support and guidance. I owe a great debt to my thesis committee---Ryan Baranowski, Chelsea Lensing, Michael Stobb, and Drew Westberg---for their patience as I struggled to complete the project in a timely manner. Ryan has an unparalleled willingness to explore research ideas with students, and his own ability to develop the intuition behind economic models has truly been such a help for me on this project and throughout the major. Chelsea's mind for empiricism has always inspired me, and her insight  has been hugely influential on my development of empirical skills in economic research. Michael Stobb is a professor like few others, and although I am not a data science major, I must admit that him and his classes often make me wish I was. In the few classes I have taken with him, I have been fortunate to glean so many skills that I am certain will be incredibly useful in the future. Most of all, I thank my primary academic advisor, Drew Westberg. I credit Drew for leading me (intentionally or otherwise) to pursue ideas that are not just intriguing, but relevant and impactful to others. More so than anything else, I thank Drew for being such a staunch personal advocate, consistently believing in my ability to succeed even when I have not been so convinced. In addition to Corinne members of my thesis committee, I thank other members of the Economics and Math faculty, including Rick Eichhorn, Brittney Miller, Jon White, Gavin Cross, and Doc Heron. I have enjoyed and learned so much in all of their classes and am certain that this project would not have come to fruition without their teaching and support. I fully believe that the College is tremendously better off because of these people.

Lastly, I thank my fellow students. I am grossly indebted to other economics students, particularly Corrine Doty, Natalie Hansen, Fiona Kilgore, Keegan Koenen, Toby Lister, and Owen Wagner. These people have consistently inspired me with their work ethic, dedication to others, and overall well-roundedness. I thank them not only for being brilliant people, but for their compassion and friendship to me and to others. I cannot imagine a better group of people to share classes, research, and life with over these past four years. In addition, I thank my coworkers in the Writing Center, many of whom I am beyond fortunate to call friends. I have never known a community so predisposed to kindness and so consistent in its ability to bring me joy.  Finally, I thank my roommate of the past four years and friend Blake for his continual support of my antics and for his comradery. Even as I look over this text---the culmination of all my academic work---I am still convinced that my greatest success at Coe has been taking the time to know and experience life with these people. Thank you.


\newpage

% \begin{center}
% 	\Large \href{https://github.com/EAPerry/seniorThesis/raw/main/Writing/Draft/main.pdf}{Click Here for Latest Version of the Paper}\\
% 	\normalsize Version: \today
% \end{center}


\section*{Preface}
\addcontentsline{toc}{section}{Preface}

Economists widely support the implementation of carbon pricing policies to reduce greenhouse gas emissions and mitigate the damages of climate change. The primary justification for carbon pricing policies, such as carbon taxes and cap-and-trade programs, is that they are efficient. Here, efficient means that these policies minimize the total cost of abatement. Alternative command-and-control policies could induce the same level of abatement, but they would likely come with a higher cost. Although this feature of environmental markets ensures the greatest net benefit for a given level of abatement, it does not make any guarantees about the distribution of these benefits. That is to say, the distributional implications of carbon pricing schemes are generally ambiguous. 

Given that equality and justice have become cornerstones of the contemporary public discourse on climate change, it should not be surprising that leading voices within the environmental community remain skeptical of carbon pricing. Contemporary carbon pricing programs take these concerns into account and typically feature some form of a ``carbon dividend'' as a way to redistribute revenue generated through carbon pricing towards those who are most exposed to climate change risk. Many jurisdictions with carbon pricing schemes do redirect sizeable portions of the revenue generated through these programs into communities that already face the worst environmental degradation. The carbon price provides individuals with an incentive to reduce their emissions, while the carbon dividend ensures a progressive distribution of the benefits from emissions reductions. 

However, less attention has been paid to the potential of carbon pricing to contribute to environmental inequalities. Many of the processes that create greenhouse gas emissions also create local air pollutants, like particulate matter, nitrogen oxides, or sulfur dioxide. Unlike the greenhouse gases covered under the carbon price, the location of this pollution matters. In this context, the carbon price might create abatement cost-effectively, but there is no guarantee that it will redistribute economic activity and the local air pollution associated with this activity in a way that avoids placing a greater air pollution burden on communities that already face greater environmental burdens. That is, because they do not have any explicit mechanism to prevent redistribution of local air pollutants, carbon pricing policies might incidentally reproduce inequitable environmental outcomes.

The objective of this paper is to analyze the implications of these carbon pricing schemes for environmental inequality. To do this, I focus on local air pollution from the electric power industry across the Western US and study how carbon pricing in this industry affects the difference in air pollution concentrations between disadvantaged and non-disadvantaged communities. 

This analysis falls into five chapters. The first two chapters provide a broad introduction to climate economics, with Chapter 1 reviewing the physical science of climate change and Chapter 2 reviewing the fundamental economic elements of climate policy design. A basic understanding of carbon pricing and emissions leakage are key to later work in the paper, so Chapter 2 takes a detailed look at both of these ideas. Using the foundational ideas from the first two chapters, Chapter 3 reviews the specific context of the analysis, reviewing local air pollution and the immediately adjacent literature. Chapter 4 outlines a novel economic model of environmental inequality associated with electric power generation. The model relies heavily on the model in \cite{weber2021dynamic}, but generalizes it into a multi-region model that can accommodate region-specific carbon prices and includes a measure of air pollution disparities analogous to the measure in \cite{hernandez2023environmental}. It features generators that make investment and operating decisions within perfectly competitive wholesale markets for electricity. Chapter 5 confronts the model from Chapter 4 with data by simulating generator investment and operating decisions over a three year period across the power grid in the Western US. I use the simulation model to measure how a carbon tax on electricity generation in California affects air pollution disparities. The simulation leads to the finding that carbon pricing does exacerbate environmental inequality, increasing the average concentration of nitrogen oxides for disadvantaged communities while decreasing the concentration for non-disadvantaged communities. Carbon pricing does not have any significant effect on disparities in sulfur dioxide or particulate matter concentrations. The mechanisms behind these simulation results demonstrate the importance of considering environmental inequalities from unilateral carbon pricing within a broader geographic context.

% \begin{abstract}
% 	Economists widely support the implementation of a carbon pricing scheme (a carbon tax or cap-and-trade program) to mitigate the damages of climate change. Despite this broad support, the effect of carbon pricing on the distribution of outdoor air pollution and the resulting disparities in air pollution exposure between communities remains understudied. This research will concentrate on studying this concern in the context of a cap-and-trade program on California's electric power industry. Methodologically, I will expand on the work of Weber (2021) by (1) explicitly modeling the ``environmental justice gap'' from , (2) considering an open economy with the potential for the redistribution of generation outside of California, and (3) potentially employing a chemical transport model to more accurately estimate air pollution exposure. \textbf{Details of results are ~forthcoming~.}
% \end{abstract}


% Chapter 1 lays the physical science foundation for climate change and climate policy. This starts by describing climate change and reviewing the evidence that not only is climate change happening, but that current climate change is attributable to humans. After establishing greenhouse gas emissions as the physical driver of climate change, the discussion shifts to contextualizing the composition and trends of greenhouse gas emissions. The chapter concludes with a discussion of the impacts of climate change. This discussion begins by walking through a set of eight major climate risks developed by the United Nations' Intergovernmental Panel on Climate Change (IPCC), and ends by describing an aggregated measure of the future climate impacts from emissions today that economists call the social cost of carbon. 

% Analogous to the first chapter, Chapter 2 lays the economic foundation for climate change and climate policy. Before describing foundational concepts in climate economics, the chapter begins by motivating the use of economic analysis in climate policy decision making. The paper investigates whether or not climate policy is necessary to curb greenhouse gas emissions through an economic lense, introducing externalities, public goods, and alternatives to policy. This proceeds by considering differences in market-based policy instruments and command-and-control instruments, and describing the use of both within environmental and climate policy. Given the significance of carbon pricing throughout the paper, a separate section considers the basic theory behind carbon taxes and cap-and-trade programs in greater detail. Finally, the chapter concludes by considering climate policy in a global context by reviewing literature on the pollution haven hypothesis, emissions leakage, and border carbon adjustments.


% Chapter 3


% Chapter 4 outlines a novel economic model of environmental inequality associated with electric power generation. The model relies heavily on the model in \cite{weber2021dynamic}, but generalizes it into a multi-region model that can accommodate region-specific carbon prices and includes a measure of air pollution disparities inspired by \cite{hernandez2023environmental}. It features generators that make investment and operating decisions within perfectly competitive wholesale markets for electricity. Although the model is more computational than analytical in nature, the chapter concludes by informally characterizing the pathways through which carbon pricing could affect air pollution disparities. \textbf{MORE ABOUT THE MODEL RESULTS HERE}

% Chapter 5 takes data to the model from Chapter 4. 

% The overarching goal of this paper is to allow these two components---climate policy and inequality---to speak in tandem with one another. Specifically, 
