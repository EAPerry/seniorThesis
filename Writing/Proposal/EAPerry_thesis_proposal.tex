\documentclass[11pt]{article}

%%%%%%%%%%%%%%%%%%%%%%%%%%%%%%%%%%%%%%%%%
% Preamble -- add your favorite packages, shortcuts, options, etc.
%%%%%%%%%%%%%%%%%%%%%%%%%%%%%%%%%%%%%%%%%

% Math formatting necessities
\usepackage{amsfonts,amssymb,amsmath,amsthm, mathrsfs}

% Page margins
\usepackage{geometry}
\geometry{top=1in,bottom=1in,left=1in,right=1in}

% Adding some better options with tables
\usepackage{array}
\renewcommand{\arraystretch}{1.15}
\newcolumntype{L}[1]{>{\raggedright\let\newline\\\arraybackslash\hspace{0pt}}m{#1}}
\newcolumntype{C}[1]{>{\centering\let\newline\\\arraybackslash\hspace{0pt}}m{#1}}
\newcolumntype{R}[1]{>{\raggedleft\let\newline\\\arraybackslash\hspace{0pt}}m{#1}}

% Custom colors
\usepackage[dvipsnames]{xcolor}
\definecolor{good_red}{RGB}{136, 0, 17}
\definecolor{good_blue}{RGB}{0, 100, 125}

% Formatting internal and external links
%\usepackage{hyperref}
\usepackage{hyperref} % If you want to see what pages we cite something
\hypersetup{
	colorlinks=true,
	linkcolor= black,
	citecolor = good_blue,
	urlcolor = good_blue
	}
\urlstyle{same}

% For images and visualizations
\usepackage{graphicx}
\usepackage{tikz, pgfplots}
\pgfplotsset{compat=1.18}
\usepackage{caption, subcaption}

% Citation management
\usepackage{natbib}
% \usepackage{harvar} % For getting some good citation styles
\bibliographystyle{aer}

% Other useful packages
\usepackage{setspace} 
\usepackage{pdflscape}
\usepackage{enumitem}
\usepackage{kpfonts} % For a better aesthetic

% Document specifics
\title{The Implications of Carbon Pricing for Environmental Inequality}
\author{Evan Perry}
\date{Last Revised: \today}


%%%%%%%%%%%%%%%%%%%%%%%%%%%%%%%%%%%%%%%%%
% Begin Document
%%%%%%%%%%%%%%%%%%%%%%%%%%%%%%%%%%%%%%%%%


\begin{document}

\maketitle
 
\begin{center}
	\emph{Do carbon pricing policies exacerbate inequalities in air pollution exposure?}
\end{center}

\renewcommand{\abstractname}{Summary}
\begin{abstract}\normalsize
\noindent 
Economists widely support the implementation of a carbon pricing scheme---either a carbon tax or cap-and-trade program---to mitigate the damages of climate change \citep{climate_letter}. 
%Despite this broad support, the effect of carbon pricing on the distribution of outdoor air pollution and the resulting disparities in air pollution exposure between communities remains understudied. 
Despite this broad support, there are remaining concerns about the potential for carbon pricing schemes to redistribute ambient air pollution in inequitable ways. Carbon pricing does not include any explicit mechanisms that would prevent, for example, substantial pollution reductions in high-socioeconomic status neighborhoods but only modest pollution reductions (or potentially even increases) in low-socioeconomic neighborhoods. This research will study the empirical validity of this concern in the context of California's electric power industry and the corresponding cap-and-trade program for greenhouse gas emissions. Methodologically, I will expand on the work of \cite{weber2021dynamic} by (i) explicitly modeling the ``environmental justice gap'' from \cite{hernandez2023environmental}, (ii) considering an open economy with the potential for the redistribution of generation outside of California, and (iii) potentially employing a chemical transport model to more accurately estimate air pollution exposure.
\end{abstract}

\doublespacing
\section*{Background \& Motivation}

Climate change threatens to dramatically alter the relationship between humanity and the environment. Decades of research have made it clear: the release of greenhouse gasses---primarily from the burning of fossil fuels---has caused the planet to warm and will continue to cause the planet to warm \citep{ipcc6_1}. Under a ``business-as-usual'' emissions scenario, \cite{carleton2022} estimate the net increase in global mortality rates attributable to climate change will be on par with current global mortality rates for all infectious diseases or all cancers by 2100. Climate change threatens to cause the extinction of unique species, with disproportionate damage to low- and middle-income countries, global declines in welfare, greater risk of extreme weather events, and the breach of climatic ``tipping points'' \citep{ipcc6_2}.

Unfortunately, the combustion processes that create anthropogenic greenhouse gas emissions often create other air pollutants as well. Many of the most common ambient (outdoor) air pollutants, like particulate matter, nitrous oxides, and sulfur dioxide, are commonly released in the same reactions that generate greenhouse gas emissions. For this reason, these are also known as ``co-pollutants.'' These co-pollutants present a serious threat to public health in their own right, causing a variety of cardiovascular and respiratory risks such as asthma, heart disease, stroke, and lung cancer. The World Health Organization estimates that ambient air pollution was responsible for 4.2 million premature deaths in just 2019 \citep{who_factsheet}. Most of these deaths occur in low- and middle-income countries, but even in the US, \cite{lelieveld2019effects} estimate that 230,000 premature deaths occur annually due to anthropogenic ambient air pollution.

To reduce greenhouse gas emissions and mitigate further damages from the climate crisis, economists widely favor a carbon pricing scheme---either a carbon tax or a cap-and-trade program.\footnote{Typically, proposals for carbon pricing do not just cover carbon dioxide emissions, but a range of other common greenhouse gasses. I use carbon pricing throughout as a shorthand for what are truly greenhouse gas emissions pricing schemes.} \cite{keohane2016markets} highlight the two primary appeals of a carbon pricing scheme: (1) under a set of common assumptions, a carbon pricing scheme produces the cost-minimizing allocation of abatement, and (2) a carbon pricing scheme creates enduring incentives for research and investment in new abatement technologies. Neither of these is guaranteed with ``command-and-control'' policies.\footnote{
	For instance, if there is substantial heterogeneity in the emissions intensity of natural gas generation, then homogenous abatement technology requirements will only achieve abatement at high costs as they do not distribute abatement preferentially on firms that have low-cost abatement technologies. A best-available technology standard, which requires all firms to adopt the best-available abatement technology, might actually give firms reason to suppress any research into new abatement technologies to avoid forcibly adopting an expensive, new abatement technology. \cite{keohane2016markets} are also careful to add several caveats to this.
} While the first of these conditions ensures the greatest net benefit to society as a whole, it does not make any guarantees about the distribution of these benefits. The equity implications of carbon pricing schemes are generally ambiguous.  

Given that equity has become a cornerstone of the contemporary public discourse on climate change, it should not be surprising that leading voices within the environmental community remain uncertain of carbon pricing's role in climate action. \cite{fischer2021green} notes that Google searches related to carbon pricing have fallen dramatically in the decade after the 2009 Waxman-Markey cap-and-trade bill and some popular Democratic presidential candidates who espoused the necessity of carbon pricing in 2016 were not nearly as vocal about carbon pricing in the 2020 election cycle. The resolution for a Green New Deal, which places social justice firmly at the center of climate policy, notably omits any mention of carbon pricing.\footnote{While there is no mention of ``carbon pricing'', ``carbon taxes'', or ``cap-and-trade'' in the resolution, the resolution does include the adoption of ``border [carbon] adjustments.'' This is some indication that the omission of carbon pricing in the resolution is a rhetorical choice more so than a policy prescription. In fact, it is commonly understood that without an accompanying carbon tax, the implementation of a border carbon adjustment would violate international trade law \citep{cosbey2019developing}.} 

Ensuring an equitable energy transition requires additional research to explore how market-based policies affect the distribution of equilibrium outcomes. Many of the distributional concerns with carbon pricing have focused on the tax burden and how to reallocate tax revenue through carbon dividends. Still, there is an emerging literature that considers instead the impact of carbon pricing on the distribution of co-pollutants. Carbon pricing derives its cost-minimizing nature by allowing markets to determine where abatement occurs, but if markets force much more abatement in high-socioeconomic status communities than low-socioeconomic status communities, carbon pricing might actually exacerbate existing disparities in air pollution exposure.

% While the economics profession still overwhelmingly supports carbon pricing, discourse amongst prominent climate economists has recently moved in a direction that suggests greater interest in ``command-and-control'' style policies to complement market-based policy instruments. Reasons for this shift in perspective include the apparent inability for market-based instruments alone to address the overlapping market failures involved in climate change \citep{stern2022economics} and, in certain contexts, small cost differences between market-based and command-and-control policies \citep{borenstein2022carbon}.

The objective of this research is to study how market-based climate policies, like a cap-and-trade program for greenhouse gasses, impact inequalities in air pollution exposure. I focus on analyzing the research question in the context of California's wholesale electricity market and the State's associated market for greenhouse gas emissions. In the remainder of this proposal, I describe the current state of the literature on the environmental justice implications of carbon pricing, highlight how the proposed research would contribute to the literature, and discuss the methodology I anticipate using to answer the research question.


\section*{Relation to the Literature}

Empirical public policy and economic research on the effect of carbon pricing schemes on the distribution of air pollution has only recently emerged. Due to data availability, much of the research so far has focused on California's cap-and-trade program. Early work developed stylized facts about the implementation of the cap-and-trade programs. \cite{cushing2018carbon} notes that even though total emissions from regulated facilities decreased in the three years after the implementation of California's cap-and-trade program, 52\% of facilities actually increased their emissions. Further, \cite{cushing2018carbon} finds that the communities with increases in co-pollutant emissions are on average poorer, less educated, and have a higher proportion of non-white residents than communities with decreases in co-pollutant emissions. In more recent study with a longer span of data and similar methods, \cite{pastor2022up} finds that the apparent inequities tied to California's cap-and-trade program still exist.

While these two studies help establish important descriptions of environmental justice outcomes in California, as \cite{hernandez2022importance} note, these do little to speak to the actual effect of the cap-and-trade program. First, their results cannot disentangle the effects of the cap-and-trade program from contemporaneous events that may cause the redistribution of co-pollutants. Pollution intensive activities are highly responsive to macroeconomic trends, and it is entirely possible that the redistribution of co-pollutants towards disadvantaged communities is a consequence of these macroeconomic trends rather than the effects of the cap-and-trade program. Second, co-pollutants are often not stagnant, but move into neighboring communities based on geography and atmospheric conditions. This means that even if the emissions of co-pollutants increases in a community, this is not sufficient information to suggest that the air pollution exposure in that community increases as well. 

\cite{hernandez2023environmental} is the first study to provide credible causal measurements of the impact of the cap-and-trade program on air pollution exposure. The authors define and measure changes in the ``environmental justice gap,'' the average difference in air pollution concentrations between disadvantaged and other communities.\footnote{California designates certain census tracts as ``disadvantaged” as a part of the CalEnviroScreen.} In contrast to \cite{cushing2018carbon} and \cite{pastor2022up}, Hernández-Cortés and Meng find evidence that California's cap-and-trade program actually reduced the environmental justice gap, by 6-10\% annually. Hernández-Cortés and Meng address the two limitations of earlier descriptive analysis by (1) using a difference-in-differences model that makes use of the staggered implementation of the cap-and-trade program to disentangle the effects of the cap-and-trade program from other contemporaneous events, and (2) embedding the predicted facility-level co-pollutant emissions within a chemical transport model that allows them to accurately measure air pollution exposure. Although Hernández-Cortés and Meng find evidence the cap-and-trade program has reduced disparities in air pollution exposure, they are also careful to emphasize that a cap-and-trade is not necessarily the sufficient to reduce these disparities. Nonetheless, their results suggest that Californians need not worry that the State's cap-and-trade program will exacerbate existing disparities in air pollution exposure. 

The previously mentioned literature studying the effects of carbon pricing on air pollution disparities has all focused on ex-post analysis of such policies. While retrospective research is vital to ensuring the success of California's cap-and-trade program going forward, the highly contextualized nature of the analysis makes the external validity of these results questionable. The econometric analysis cannot describe any underlying mechanisms that produce the measured causal effects, and without a clear understanding of \emph{how} California's cap-and-trade program helped to close disparities in air pollution exposure, we cannot anticipate the effects of similar policies applied elsewhere. 

\cite{weber2021dynamic} offers, to my knowledge, the first and only ex-ante model that studies how carbon pricing in California affects the spatial redistribution of co-pollutants. The model focuses on the State's electric power industry and follows in the spirit of related structural, industrial-organization models \citep[e.g.,~][]{gowrisankaran2022policy, abito2022role}. Although Weber focuses primarily on the total welfare effects of the redistribution of co-pollutants, her results suggest that counties with more ``disadvantaged” communities appear to also see greater reductions in co-pollutant emissions. These findings pair well with those in \cite{hernandez2023environmental}, although Weber focuses on only power plants and Hernández-Cortés and Meng focus on all regulated facilities except power plants.

This research project will focus on extending the model and empirical techniques used in \cite{weber2021dynamic} into the context of an open economy with flexible imports and exports of electricity. Although this may seem to be an unnecessary generalization of the model, the incomplete nature of California's unilateral carbon pricing scheme opens up wider channels for co-pollutant redistribution. In context, the incompleteness of the carbon market is important as carbon pricing policies have the potential to not only redistribute co-pollutants within California, but to increase and redistribute co-pollutants outside the State as well. The case of California electricity is especially interesting as much of the electricity generation that occurs outside of the state is ``dirty” relative to California power. Weber diligently notes that emissions leakage is not a significant concern in California's electricity market, an assumption that has since been corroborated \citep{burtraw2018}. Still, the extension of the model into the context of an incomplete carbon market is valuable as this can be a substantial concern for other emissions-intensive, trade-exposed goods like cement, aluminum, and steel \citep{fowlie2022mitigating}. Consequently, this research is pertinent to a broader literature concerned with how carbon pricing and could reinforce cross-jurisdictional inequalities. 

If time and resources allow, this paper will also address a second limitation of the results in \cite{weber2021dynamic}: the modeling of air pollution exposure. Weber uses an air pollution damages model that is appropriate for studying total welfare effects, but admits that ideally the study would embed the co-pollution estimates within a chemical transport model to estimate the changes in air pollution exposure of affected communities \citep[the approach in~][]{hernandez2023environmental}. The chemical transport models necessary for this analysis are available but are computationally expensive.


\section*{Methodology}

The model will largely follow from \cite{weber2021dynamic}, but with two primary differences.\footnote{The model Weber uses is itself mostly based on the models in \cite{cullen2015} and \cite{cullen2017}.} First, my model will place equity implications at the forefront of the analysis by explicitly incorporating a measure of disparities in air pollution exposure. My measure will be similar to the ``environmental justice gap'' measure in \cite{hernandez2022importance}. Second, my model will consider an open economy. Weber's model considers a closed economy which implicitly eliminates any potential emissions leakage and cross-jurisdictional co-pollutant redistribution. 

The model in \cite{weber2021dynamic} focuses on generator-level investment and production decisions. In an initial period, each generator has the opportunity to make an investment that will improve its efficiency. The efficiency of the generator influences the marginal costs of generation both by reducing fuel costs and reducing the cost from the carbon tax. In all subsequent periods, the generator must decide whether or not it will operate. This decision is complicated by ramping costs---the extra cost incurred when operating a generator that was not operating in the previous period. Generation is sold on the wholesale electricity market, which is assumed to be competitive. Demand is completely inelastic as end-users will not observe prices in the wholesale market, but is unknown to the generators in advance. Leveraging a result from \cite{cullen2015}, Weber establishes that the solution to the dynamic problem corresponds with a cost-minimization problem. 

With this model, Weber establishes two channels for carbon pricing to change the spatial distribution of co-pollutants: (1) by changing the relative order of generators along the supply curve, and (2) by changing the co-pollutant intensity of generators by inducing efficiency-improving investments. While these same two channels will be present in my own version of the model, my model will also allow for the redistribution of co-pollutants attributable to the incompleteness of the carbon market.

To incorporate an open economy and allow for cross-jurisdictional redistribution, I will employ elements from other models of emissions leakage in California's electricity sector. This is closest to the model in \cite{fowlie2021border}. \cite{fowlie2009incomplete} may also be useful, although this model considers a wholesale electricity market with market power. 

Using the model, I will focus on characterizing the channels through which co-pollutant redistribution can occur. Different from Weber, this characterization will explicitly model disparities in air pollution exposure. This model will clarify how the carbon pricing scheme leads to the redistribution of air pollutants, and hopefully develop some guidelines that describe when we should expect carbon pricing to exacerbate existing disparities in air pollution exposure. 

The data and empirical strategy should be similar to Weber's. In her paper, data come from SNL, a proprietary data product created by S\&P Global Marketplace. However, all of the data on SNL is collected from publicly available sources and are avialable elsewhere. I include the primary data sources required to calibrate and simulate the model below: 
\begin{itemize}
	\item \href{http://oasis.caiso.com/mrioasis/logon.do}{California Independent System Operator Open Access Same-time Information System (CAISO OASIS)}: Hourly wholesale electricity prices at pricing nodes in and around California.  
	\item \href{https://campd.epa.gov/data/bulk-data-files}{EPA Clean Air Markets Program Data}: Hourly generation and emissions data for individual generators. 
	\item \href{https://www.epa.gov/egrid}{Emissions \& Generation Integrated Resource Database (eGRID)}: Stable generator characteristics related to generation, greenhouse gas emissions, and criteria air pollutants.
\end{itemize}
I anticipate following model calibration and simulation steps similar to those that Weber uses. 

\section*{Logistics}

I will have a full draft of the paper done by March 10, 2023. My hope is that the faculty members on my thesis committee will have time over the month of March to provide feedback on this draft. After working through revisions with the faculty members on my committee, I anticipate having a finished draft by mid-April. I also anticipate defending my thesis in mid-April, with the exact date dependent on faculty availability. I will make a habit of uploading data (to the extent possible), code, and my writing to the GitHub repository \href{https://github.com/EAPerry/seniorThesis}{linked here}. 

\section*{Thesis Outline}

Attached below is a rough outline for the paper. The first three sections are written at an introductory level and provide broad grounding for the central research question. The fourth section, ``Emissions Pricing in California's Electric Power Industry,'' provides context for the specific environment I choose to model in and test empirically. The remaining sections cover the bulk of the original research and are written more with an audience experienced in economics in mind than the previous sections. I expect the finished draft will be in the range of 100--130 pages (formatted according to the Library's standards).

\begin{enumerate}
	\item Introduction
	\item Background on Climate Change \& Ambient Air Pollution
	\begin{enumerate}
		\item The Earth is Warming (and it's Our Fault)
		\item Greenhouse Gas Emissions: Structure \& Trends
		\item Risks \& Impacts of Climate Change
		\item Review of Ambient Air Pollution
	\end{enumerate}
	\item Designing Climate Policy
	\begin{enumerate}
		\item A Case for Economic Analysis in Climate Policy Design
		\item An Economic Motivation for Climate Policy
		\item The Structure \& Scope of Environmental Policy
		\item Carbon Pricing \& Environmental Markets
		\item Incomplete Carbon Markets
		\item Distributional Considerations in Market-Based Policy Design
	\end{enumerate}
	\item Emissions Pricing in California's Electric Power Industry
	\begin{enumerate}
		\item Introduction to California's Wholesale Electricity Market
		\item Emissions Pricing 
		\item Background on Air Pollution Disparities
	\end{enumerate}
	\item A Model of Emissions Pricing \& Inequities in Air Pollution Exposure
	\begin{enumerate}
		\item Model Environment
		\item The Generator's Problem
		\item Equilibrium Characterization
		\item Channels for Increased Disparities in Air Pollution Exposure
	\end{enumerate}
	\item Model Application
	\begin{enumerate}
		\item Data
		\item Empirical Strategy \& Model Calibration
		\item Model Diagnostics
		\item Results
		\item Discussion \& Limitations
	\end{enumerate}
	\item Conclusion
	\item References
	\item Appendices (As Needed)
\end{enumerate}

\bibliography{References}

\end{document}